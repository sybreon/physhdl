
\documentclass[a4paper,11pt]{book}
\usepackage[T1]{fontenc}
\usepackage[utf8]{inputenc}
\usepackage{lmodern}
\usepackage{setspace}
\usepackage{color}
\usepackage{xcolor}
\usepackage{fancyhdr}

\usepackage{listings}
% \usepackage{fancyvrb}
\lstset{basicstyle=\small\ttfamily,columns=fixed,numbers=right,frame=Tb}
\renewcommand{\lstlistingname}{HDL}

% \usepackage{caption}
% \DeclareCaptionFont{white}{\color{white}}
% \DeclareCaptionFormat{hdl}{\colorbox{black}{\parbox{\textwidth}{#1#2#3}}}
% \captionsetup[lstlisting]{format=hdl,labelfont=white,textfont=white}

\usepackage{hyperref}
% \usepackage{tocloft}

\title{Physical HDL: A Graphical Approach}
\author{Shawn Tan\\ \tiny{PhD(Cantab), CEng MIET}}

\begin{document}

\frontmatter
\maketitle
\tableofcontents

\chapter*{Preface}

I started writing this book while I was teaching a subject called Digital Systems \& HDL.
Through the years, I have come across a myriad of texts that cover the topic but never quite in the way that is useful.
I have noticed that while students and young engineers are generally capable of learning the language on their own, 
they lack the necessary capability to link the language to the physical hardware implementation.
That is what motivated me to write this book.

% What is this book about and how is it different from other texts?
% It will be easier to describe what this book isn't rather than is.

Many people typically approach HDL based design just like they approach software programming but this is a \emph{disaster}.
While the two may look similar, they are fundamentally different and using one like the other will invariably result in badly designed systems.

This is not a book on hardware description languages (HDL).
There are many texts on the various HDLs available, particularly the popular ones such as Verilog and VHDL.
Those who are interested to learn the intricacies and subtleties of each language should refer to those established texts.
In Part \ref{PART:HDL} of this book, I will merely review a limited sub-set of these languages, for those who may wish to learn just the bare essentials of the language.

Similarly, this is not a book on digital systems design.
There are many texts on digital systems design, at both the undergraduate and graduate levels.
However, these texts are typically general in nature and do not necessarily focus on HDL based designs.
In Part \ref{PART:SYS} of this book, I will graphically show how each component of a digital system can be designed using HDL with cook-book styled examples.

Neither is this a book on chip design.
There are other texts that are more detailed on this topic, specifically those provided by tool vendors that are specific to their individual design processes.
In Part \ref{PART:FASM} of this book, I will set down a general design flow that is used for digital systems design using HDL.
This is essential to our understanding of where HDLs feature in the larger scheme of things.


I'm not saying that I know the best way to do things, but this is the way that I do things and it seems to work.
Therefore, I decided to write a book on this subject to share my thoughts, ideas, and experiences with the world.

I hope that it will be of some use to someone out there.


\mainmatter
\onehalfspace

\part{Designing the System}
\chapter{Standard Flow}
\section{Design Entry}
\section{Functional Simulation}
\section{Synthesis}
\section{Functional Verification}
\section{Place \& Route}
\section{Timing Verification}
\section{Tape Out}


\chapter{My Way}
\section{Top Down}
\section{Bottom Up}

\part{Reviewing the Language}
\chapter[Verilog]{Verilog Hardware Description Language}
\section{Data Types}
\section{Data Operators}
\section{Control Structures}
\section{Data Structures}

\chapter[VHDL]{Very High Speed Integrated Circuit Hardware Description Language}
\section{Data Types}
\section{Data Operators}
\section{Control Structures}
\section{Data Structures}

\part{Connecting the Dots}

% Copyright (C) 2012, Shawn Tan <shawn.tan@sybreon.com>
% Permission is granted to copy, distribute and/or modify this document
% under the terms of the GNU Free Documentation License, Version 1.3
% or any later version published by the Free Software Foundation;
% with no Invariant Sections, no Front-Cover Texts, and no Back-Cover Texts.
% A copy of the license is included in the section entitled "GNU
% Free Documentation License".

\chapter{Combinatorial Logic}

Combinatorial logic, or asynchronous logic, is the basic of all digital logic systems.
This logic is typically made up of a \emph{sea} of gates that provide specific outputs in response to changes in the inputs.


\section{Basic Gates}

The fundamental unit of all this is the logic gate, which is capable of producing a discrete logic output of either 1 or 0.
Therefore, it is necessary to learn how to infer and instantiate logic gates as needed.

\subsection{Positive/Negative Logic}
The basic logic gates available in all digital systems are the: NOT, AND, OR, XOR, NAND, NOR, XNOR.

\subsection{Inference/Instantiation}
There are two ways of designing combinatorial logic systems: inference and instantiation.

\section{Logic Primitives}

Standard combinatorial logic circuits are made up of a combination of one or more logic primitives.
The basic logic primitives are made up of logical NOT, AND, OR, and XOR.

\subsection{NOT}

% Show code, show output for single gate.
% Show code, show output for cascade/parallel combinations.
% ditto with the rest.

\subsection{AND/NAND}

\subsection{OR/NOR}

\subsection{XOR/XNOR}

\section{Encoders}
\section{Multiplexers}

% \lstinputlisting[language=verilog,caption=4-input Mux]{../rtl/vlog/mux4.v}
% \lstinputlisting[language=vhdl,caption=4-input Mux]{../rtl/vhdl/mux4.vhd}


\chapter{Synchronous Logic}
\section{Flip Flops}
\section{Counters}
\subsection{Binary}
\subsection{Ring}
\subsection{Gray}
\section{Shift Registers}
\subsection{Linear Feedback Shift Register}

\chapter{Integer Arithmetic}
\section{Adder}
\subsection{Addition}
\subsection{Subtraction}
\subsection{Carry}
\section{Multiplier}
\section{Bit Logic}
\subsection{Masking}
\subsection{Set Bits}
\subsection{Clear Bits}
\subsection{Toggle Bits}

\chapter{Finite State Machine}
\section{Moore Machine}
\section{Mealy Machine}

\part{Simplifying the Circuit}

\part{Verifying the Works}

\chapter{Simulation Constructs}
\chapter{Finshing Up}

\backmatter
% \tiny
% \include{pre/fdl-1.3}

\end{document}
