% Copyright (C) 2012, Shawn Tan <shawn.tan@sybreon.com>
% Permission is granted to copy, distribute and/or modify this document
% under the terms of the GNU Free Documentation License, Version 1.3
% or any later version published by the Free Software Foundation;
% with no Invariant Sections, no Front-Cover Texts, and no Back-Cover Texts.
% A copy of the license is included in the section entitled "GNU
% Free Documentation License".

\chapter{Synchronous Logic}

Synchronous logic is the bedrock of almost all modern digital systems.
Everything that happens withing such a system is synchronised to a clock.

\section{Flip Flops}

The basic circuit element of a synchronous logic system is a flip-flop.
A flip-flop is a device that will only change its output state in relation to a clock trigger.

\subsection{Basic Flip Flop}

% DFF
% DFF into a TFF/JKFF

\subsection{Clock Enable}
\subsection{Synchronous Reset}
\subsection{Asynchronous Reset}
\subsection{Reset \& Clock Enable}

\section{Counters}
\subsection{Binary}
\subsection{Ring}
\subsection{Gray}
\section{Shift Registers}
\subsection{Linear Feedback Shift Register}
