% Copyright (C) 2012, Shawn Tan <shawn.tan@sybreon.com>
% Permission is granted to copy, distribute and/or modify this document
% under the terms of the GNU Free Documentation License, Version 1.3
% or any later version published by the Free Software Foundation;
% with no Invariant Sections, no Front-Cover Texts, and no Back-Cover Texts.
% A copy of the license is included in the section entitled "GNU
% Free Documentation License".

\chapter{Combinatorial Logic}

Combinatorial logic, or asynchronous logic, is the basic of all digital logic systems.
This logic is typically made up of a \emph{sea} of gates that provide specific outputs in response to changes in the inputs.


\section{Basic Gates}

The fundamental unit of all this is the logic gate, which is capable of producing a discrete logic output of either 1 or 0.
Therefore, it is necessary to learn how to infer and instantiate logic gates as needed.

\subsection{Positive/Negative Logic}
The basic logic gates available in all digital systems are the: NOT, AND, OR, XOR, NAND, NOR, XNOR.

\subsection{Inference/Instantiation}
There are two ways of designing combinatorial logic systems: inference and instantiation.

\section{Logic Primitives}

Standard combinatorial logic circuits are made up of a combination of one or more logic primitives.
The basic logic primitives are made up of logical NOT, AND, OR, and XOR.

\subsection{NOT}

% Show code, show output for single gate.
% Show code, show output for cascade/parallel combinations.
% ditto with the rest.

\subsection{AND/NAND}

\subsection{OR/NOR}

\subsection{XOR/XNOR}

\section{Encoders}
\section{Multiplexers}

% \lstinputlisting[language=verilog,caption=4-input Mux]{../rtl/vlog/mux4.v}
% \lstinputlisting[language=vhdl,caption=4-input Mux]{../rtl/vhdl/mux4.vhd}
